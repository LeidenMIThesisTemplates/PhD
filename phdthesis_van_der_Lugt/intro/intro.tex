%! TEX root = intro.tex
\documentclass[../phdthesis]{subfiles}

\begin{document}
\ifSubfilesClassLoaded{\frontmatter}{}
\chapter{Introduction}
This setup allows you to work on chapters as standalone TeX-documents.
In this case you could compile the file chapter1.tex in the chapter1 directory to only compile the first chapter.
You could also compile phdthesis.pdf in the main document to compile everything.
This is useful if your thesis becomes large.

It is possible to put references between chapters by prepending the references with the string `M-'.
To update all labels compile the main document.
It's a bit hacky but this is how I got it to work.
Example: Chapter \ref{M-chapter1} makes a reference to the chapter1 document.

Also, in the file preamble.sty you'll find a line that you can uncomment in order to compile all tex-files in the 17x24mm format, with crop marks on A4 paper.
This can be used when you want to send your thesis to the printer.

\begin{definition}
	This style uses gray boxes for theorems, definitions, et cetera.
\end{definition}
\begin{theorem}<theorem1>
	Labels inside theorem boxes should be defined by putting them between \(<>\).
	I couldn't get it to work with the \textbackslash label command.
\end{theorem}
Reference to Theorem \ref{theorem1}.

\begin{theorem}(Description of theorem)
	Descriptions of theorems can be added by putting them in between ().
\end{theorem}

\begin{theorem}>4.5.6<
	It is also possible to change the number of a theorem by putting them between \(><\). This is useful if you want to show one of your theorems in your introduction.
\end{theorem}



\ifSubfilesClassLoaded{
	\printbibliography
}{}
\end{document}
